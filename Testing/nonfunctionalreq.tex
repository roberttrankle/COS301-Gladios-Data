
	\section{Non-functional Requirements}
		\subsection{Requirements Tested}

			\subsubsection{Performance}
			\subsubsection{Scalability}
			\subsubsection{Integrity}
			\subsubsection{Accessibility}
			
	
			\subsection{Evaluation and Comments}

			%Evaluation table goes here
			
			%Comments
			\subsubsection{Performance}
			The key functionalities tested with respect to performance were speed, concurrency and response rate. The application doesn't take time to connect to
			the Aruba server and so does the response rate, that is , the rate between logging in to Aruba server and get the location back.
			
			\subsubsection{Scalability}
			Capability of the system: Smooth or rather fast uniterrupted  server (i.e Aruba) connection, however, the application cannot get all the MAC Addresses of
			the connected devices. \\
			The ability to handle multiple users: The application makes use of multiple threads structure to handle multiple user access, so each new device has it's own thread.

			\subsubsection{Integrity}
			There is no integrity in the application, connection to Aruba server is not authorized. The application sets Aruba SSL validation to false, which provides unauthorized
			access to Aruba server, and thus there's no integrity in the application.

			\subsubsection{Accessibility}
			Accessibility testing: The main test was to test/determine whether individuals with disabilities can use the appllication, however, that's in the Access module as the Data
			module is for data streaming.

			\subsubsection{Security}
			As already stated, the application sets the SSL certification validation to false, which provides unauthorized and not-so secure access to the Aruba server.
